\documentclass[10pt,letterpaper]{article}
\usepackage[utf8]{inputenc}
\usepackage{amsmath}
\usepackage{amsfonts}
\usepackage{amssymb}
\usepackage{graphicx}
\usepackage{bbm}
\usepackage{listings}
\usepackage{url}
\usepackage{amsopn,amssymb,thmtools,thm-restate}
\usepackage{algorithm}
\usepackage{algorithmic}
\usepackage{amsopn,amssymb,thmtools,thm-restate}
\usepackage{amsthm}
\usepackage{algorithm}
\usepackage{algorithmic}

\DeclareMathOperator{\argmax}{arg\,max}
\DeclareMathOperator{\argmin}{arg\,min}
\DeclareMathOperator{\argsort}{arg\,sort}
\newcommand{\tree}{\mathcal{T}}
\newcommand{\xopt}{x^*}
\newcommand{\robot}{R}
\newcommand{\obst}{F}
\newcommand{\obj}{B}
\newcommand{\cspace}{C}
\newcommand{\state}{s}
\newcommand{\action}{a}
\newcommand{\statespace}{S}
\newcommand{\actionspace}{A}
\newcommand{\traj}{\tau}
\newcommand{\op}{\mathfrak{o}}
\newcommand{\Op}{\mathcal{O}}
\newcommand{\inp}{\alpha}
\newcommand{\policy}{\pi}
\newcommand{\plpolicy}{\pi_{\text{pl}}}
\newcommand{\disc}{\delta}
\newcommand{\discspace}{\Delta}
\newcommand{\cont}{\kappa}
\newcommand{\contspace}{K}
\newcommand{\solop}{\contspace^*_s }
\newcommand{\Dpl}{\mathbf{D}_\text{pl}}
\newcommand{\Drl}{\mathbf{D}_\text{RL}}
\newcommand{\opol}{\pi_{i}}
\newcommand{\aw}{a^{\op}}
\newcommand{\hQ}{\hat{Q}_{\alpha}}
\newcommand{\regQ}{\mathcal{R}(\hQ)}
\newcommand{\Ppl}{P_{pl}}
\newcommand{\Ppi}{P_{\pi}}
\newcommand{\good}{\epsilon}
\newcommand{\prob}{\omega}
\newcommand{\probfeature}{\phi}
\newcommand{\admon}{{\sc AdMon}}
\newcommand{\gtamp}{G-TAMP}
\newcommand{\ddpg}{{\sc ddpg}}
\newcommand{\ppo}{{\sc ppo}}
\newcommand{\gail}{{\sc gail}}
\newcommand{\region}{\mathcal{R}}
\newcommand{\motion}{\tau}
\newcommand{\pickconf}{c_{pick}}
\newcommand{\placeconf}{c_{place}}
\newcommand{\initconf}{c_{init}}

\newcommand{\trj}{operator sequence}
\newcommand{\tr}{sequence}
\newcommand{\trjs}{operator sequences}
\newcommand{\trs}{sequences}

\newcommand{\workspace}{\mathcal{W}}


\author{Beomjoon Kim}
\title{Reusable sub-goal based Monte Carlo planning algorithm}
\begin{document}
\maketitle
\section{Introduction}
\emph{Motivation:}
\begin{itemize}
\item In many complex robot planning problems,
parts of the solutions of the entire planning problem may be known in advance 
because they were learned, planned, or specified by a human.
\item For instance, for a cooking problem, a human might hint the planner that the object must be
washed before placed on a stove, or a hierarchical planner might determine that
a swept volume to the object should be cleared first, for the simple problem of
picking an object up.
\item We assume that these hints are represented as a sequence of 
sub-goals that define the associated sub-planning problems that, 
if you can find feasible solutions
to these sequence of sub-planning problems, then you can find a feasible solution
to the entire problem.
\item The sub-goals allow us to shorten the planning horizon, and 
focus the sampling effort on manipulating a particular set of objects
in achieving the immediate sub-goal. Moreover, if we have sub-goals for
a smaller but recurring planning problem, then we may re-use the same
sub-goals whenever the smaller planning problem needs to be solved in a
larger planning problem.
\item For example, in many robot mobile-manipulation domains, 
the navigation-among-movable-obstacles (NAMO)
problem constantly arise when the robot tries to pick-and-place (or push) an object
to clear obstacles out of the swept volume. If we can specify sub-goals for
this problem, then any planning problem that requires solving NAMO as a sub-problem
would be more efficient.
\item We do not make assumptions about where the sub-goals come from, but we require
them to be such that achieving one sub-goal do not interfere with the feasibility of
 achieving the subsequent sub-goals, but it affects their optimality.
\item Given our assumption on the sub-goals, one might naively 
try to solve the entire problem by solving
sub-problems independently. However, not all feasible solutions 
are equal: some solutions are better than the others, and solving
them independently and locally is likely to yield a feasible yet poor solution.
Our goal is to produce a solution that is not only feasible, but globally
optimal.
\end{itemize}

Planning algorithm
\begin{itemize}
\item We propose an anytime planning algorithm that, given a sequence of 
sub-goals and a reward function,
computes a plan that optimizes the sum of the rewards along the trajectory
for a deterministic planning environment with continuous action space.
\item Our algorithm is a Monte Carlo planning algorithm that updates the estimate  of
its value function based on roll-outs of trajectories.
\item Traditionally, the Monte Carlo planning has been applied to Markov Decision Processes with
discrete action spaces.
\item There are some algorithms for continuous action spaces, such as HOOT and SOOP,
which places a continuous-armed bandit agent, Hierarchical Optimistic Optimization (HOO), 
in the case of MDP,  and black-box function optimizer, 
Simultaneous Optimistic Optimization (SOO), in the case of deterministic 
environment, at each node, much like how
UCB agent is placed at each node for UCT to solve an MDP with discrete action space. 
They do not work for high dimensions. 
\item The performance of the optimization-agent at
each node heavily influences the performance of the planning algorithm.
\item We propose a new black-box function optimization algorithm that can work in
high dimensions, and analyze the regret.
\item To make use of the sub-goals, we also propose a 
root-switching technique that searches for the solution to the sub-goal from the root,
but once found, switches the root to the resulting goal node, and work towards the next goal.
Once we found a feasible solution, we switch the root back to the original root node
in order to find a better solution, while keeping all of the past roll-out branches.
\end{itemize}

Format of the sub-goals.
\begin{itemize}
\item The sub-goals are specified using the predicates used in operator preconditions
 or the predicate used to define the goal. This is based on the observation that
the preconditions naturally define the associated sub-planning problem that needs
to be solved if the associated operator is to be used in the entire planning problem.
\end{itemize}

Using our algorithm
\begin{itemize}
\item Our algorithm can be used as a stand-alone algorithm, or can be integrated
with a classical task-planner which will generate sub-goals
for our planner; but exact usage is out of scope of this paper.
\end{itemize}


Experiments
\begin{itemize}
\item Our hypotheses are: (1) using sub-goals reduce the planning time (2) our algorithm
can globally optimize the reward function, (3) Our algorithm is better than existing
Monte Carlo planning algorithms, and (4) there exists larger problems that can
benefit from sub-goal specification of recurring smaller sub-problems.
\item For (1), compare my algorithm with and without root switching
\item For (2), plot a reward curve vs time
\item For (3), compare with SOOP, HOOT, and stripstream(?)
\item For (4), show how packing and NAMO sub-goals are re-used in the
bigger problem
\end{itemize}

\bibliographystyle{IEEEtran}
\bibliography{references}
\end{document}
